\documentclass[a4paper, 12pt]{article}
\include{includes/packages}
\newcommand{\Title}{Отчет о выполнении лабораторной работы}
\newcommand{\TaskType}{лабораторная работа}
\newcommand{\SubTitle}{по дисциплине <<Большие данные: инструменты и технологии>>}
\newcommand{\LabTitle}{Spark} 
\newcommand{\Faculty}{<<Информатика и системы управления>>}
\newcommand{\Department}{<<Компьютерные системы и сети (ИУ-6)>>}
\newcommand{\AuthorFull}{Козлов Владимир Михайлович}
\newcommand{\Author}{Козлов В.М.}
\newcommand{\Teacher}{}
\newcommand{\group}{ИУ6-13М}
\newcommand{\Year}{2025}
\newcommand{\Country}{Россия}
\newcommand{\City}{Москва}

\newcommand{\UpperFullOrganisationName}{Министерство науки и высшего образования Российской Федерации}
\newcommand{\ShortOrganisationName}{МГТУ~им.~Н.Э.~Баумана}
\newcommand{\FullOrganisationName}{Федеральное государственное автономное образовательное учреждение высшего образования\newline <<Московский государственный технический университет имени Н.Э.~Баумана (национальный исследовательский университет)>> (\ShortOrganisationName)}

\include{includes/colors}
\include{includes/styles}
\include{includes/listings}
\begin{document}
\input{includes/titlepage}
\pagebreak
\tableofcontents
\newpage
% Основная часть --------------------------------------------------------------------------------------------
\section*{Задание}
\addcontentsline{toc}{section}{Задание}
\subsection*{Цель}
Целью лабораторной работы является приобретение навыков работы со Spark.
\section{Ход работы}
Для выполнения использовался тот же docker-образ, что и для ЛР, так как там уже предусмотрен spark.

В качестве текста было вабрано начало НИР прошлого семестра по теме "Распределение вершин графа по гетерогенным хранилищам"

Для начала читается текст файла по строкам командой:
\begin{lstlisting}[language=python]
    rdd = sc.textFile("/workspace/article.txt")
\end{lstlisting}

В качестве предобработки сначала строки очищаются от whitespace-символов по краям, преобразуются в нижний регистр, затем строки разделяются по пробелам на слова; далее отфильтровываются лишь слова из букв (в том числе и через тире) длиной более 3 символов с помощью регулярного выражения:
\begin{lstlisting}[language=python]
words = (rdd
    .map(lambda x: x.strip())
    .map(lambda x: x.lower())
    .flatMap(lambda x: x.split()) # [“word1 word2”, ...] -> [“word1”, “word2”]
    .filter(lambda x: re.match(r'^[A-Za-zА-Яа-я-]{4,}$', x))
)
\end{lstlisting}

Далее подготавливается промежуточный RDD indexed, формируемый добавлением каждому элементу индекс; в результате вместо каждого элемента получается пара из индекса и элемента:
\begin{lstlisting}[language=python]
indexed = (words
    .zipWithIndex()
    .map(lambda x: (x[1], x[0]) # [(word, i), ...] -> [(i, word), ...]
)
\end{lstlisting}

Для биграмм формируется другой RDD shifted, состоящий из тех же элементов indexed, у которых индекс смещён на 1:
\begin{lstlisting}[language=python]
shifted = (indexed
    .map(lambda x: (x[0] - 1, x[1]))
)
\end{lstlisting}

Далее, чтобы получить биграммы, промежуточный indexed и сдвинутый shifted группируются по ключу, которым выступают индексы, в результате чего и получается объединить все подряд идущие слова по парам; далее результат объединения превращается в одну строку из двух слов и отбрасываются  индексы:
\begin{lstlisting}[language=python]
bigrams = (indexed
    .join(shifted) # -> [(i, [“word1”, “word2”]), ...]
    .mapValues(lambda words: " ".join(words)) # -> [(i, “word1 word2”), ...]
    .map(lambda x: x[1]) # -> [“word1 word2”, ...]
)
\end{lstlisting}

Аналогичным образом для триграмм создаётся ещё один RDD со сдвигом индекса на 2 от первоначального значения:
\begin{lstlisting}[language=python]
shifted2 = (indexed
    .map(lambda x: (x[0] - 2, x[1]))
)
\end{lstlisting}

После чего аналогичным с биграммами образом объединяются сначала первые два RDD, затем с результатом объединяется ещё и третий RDD, благодаря чему получаются триграммы:
\begin{lstlisting}[language=python]
trigrams = (indexed
    .join(shifted)
    .mapValues(lambda words: " ".join(words))
    .join(shifted2)
    .mapValues(lambda words: " ".join(words))
    .map(lambda x: x[1])
)
\end{lstlisting}

Далее объявляется функция для подсчёта количества и сортировки по возрастанию:
\begin{lstlisting}[language=python]
def counted(rdd, ascending=False):
    """Подсчёт количества элементов и сортировка: [(item, num), ...]"""
    return (rdd
        .map(lambda x: (x, 1))# добавляем 1 для суммы
        .reduceByKey(lambda a, b: a + b)# складываем
        .sortBy(lambda x: x[1], ascending=ascending) # сортировка по убыванию
    )
\end{lstlisting}
И для получения требуемого результата выводятся 20 самых часто встречаемых биграмм и триграмм:
\begin{lstlisting}[language=python]
print(counted(bigrams).take(20))
print(counted(trigrams).take(20))
\end{lstlisting}
\lstinputlisting[caption=Полный скрипт ЛР, language=python]{../scripts/lab2.py}
% -------------------------------------------------------
\newpage
\section{Вывод}
В ходе лабораторной работы приобретены навыки работы со Spark.
%-----------------------------------------------------------------------------------------
\newpage
\section*{Приложение 1}
\lstinputlisting[caption=Текст, style=text]{../scripts/lab2_text.txt}
\newpage
\section*{Приложение 2}
\begin{table}[H]
\centering
\begin{tabular}{|c|c|}
\hline
\textbf{Диграммы} & \textbf{Триграммы} \\ 
\hline
распределения вершин (2) & между которые могут (2) \\
между которые (2) & алгоритм роста графа (2) \\
распределение графа (2) & распределения вершин графа (2) \\
общей мощности (2) & edge matching выбор (2) \\
алгоритм роста (2) & современные графовые базы (1) \\
роста графа (2) & таких социальные рекомендательные (1) \\
вершин графа (2) & реальном времени мере (1) \\
основной операцией (2) & распределения требующих полного (1) \\
вершин между (2) & предложений комбинированию подходов (1) \\
graph growing (2) & реальной эксплуатации графовых (1) \\
edge matching (2) & максимизации эффективности этой (1) \\
функция улучшения (2) & чтобы минимизировать общую (1) \\
пока будет (2) & первому также происходит (1) \\
также стоит (2) & процессе порядок найденное (1) \\
которые могут (2) & случайное затем случайное (1) \\
matching выбор (2) & метод прост эффективен (1) \\
современные графовые (1) & приводит большему общему (1) \\
сталкиваются фундаментальной (1) & учитывает вершины изначального (1) \\
распределение становится (1) & если противном выбранная (1) \\
операцией является (1) & обмена вершин между (1) \\
\hline
\end{tabular}
\caption{Частотный анализ диграмм и триграмм}
\end{table}
\end{document}
